\documentclass[10pt, portrait]{article}
\usepackage[scaled=0.92]{helvet}
\usepackage{calc}
\usepackage{multicol}
\usepackage{ifthen}
\usepackage[a4paper,margin=3mm,landscape]{geometry}
\usepackage{amsmath,amsthm,amsfonts,amssymb}
\usepackage{color,graphicx,overpic}
\usepackage{hyperref}
\usepackage{newtxtext} 
\usepackage{enumitem}
\usepackage[table]{xcolor}
\usepackage{mathtools}
\usepackage{nicematrix}
% for drawing diragrams/graphs
\usepackage{tikz}
\usetikzlibrary{arrows.meta}
\usetikzlibrary{calc}
\setlist{nosep}

% ADDITIONAL USEFUL PACKAGES:
% for matrices

% for relations
\usepackage{cancel}
\usepackage{ mathrsfs }
% for including images
\graphicspath{ {./images/} }


\pdfinfo{
  /Title (MA1101R.pdf)
  /Creator (TeX)
  /Producer (pdfTeX 1.40.0)
  /Author (Jovyn)
  /Subject (MA1101R)
  /Keywords (MA1101R, nus,cheatsheet,pdf)}

% Turn off header and footer
\pagestyle{empty}

\newenvironment{tightcenter}{%
  \setlength\topsep{0pt}
  \setlength\parskip{0pt}
  \begin{center}
}{%
  \end{center}
}

% redefine section commands to use less space
\makeatletter
\renewcommand{\section}{\@startsection{section}{1}{0mm}%
                                {-1ex plus -.5ex minus -.2ex}%
                                {0.5ex plus .2ex}%x
                                {\normalfont\large\bfseries}}
\renewcommand{\subsection}{\@startsection{subsection}{2}{0mm}%
                                {-1explus -.5ex minus -.2ex}%
                                {0.5ex plus .2ex}%
                                {\normalfont\normalsize\bfseries}}
\renewcommand{\subsubsection}{\@startsection{subsubsection}{3}{0mm}%
                                {-1ex plus -.5ex minus -.2ex}%
                                {1ex plus .2ex}%
                                {\normalfont\small\bfseries}}%
\renewcommand{\familydefault}{\sfdefault}
\renewcommand\rmdefault{\sfdefault}
%  makes nested numbering (e.g. 1.1.1, 1.1.2, etc)
\renewcommand{\labelenumii}{\theenumii}
\renewcommand{\theenumii}{\theenumi.\arabic{enumii}.}
\renewcommand\labelitemii{•}
%  highlighting for math
\newcommand{\mathcolorbox}[2]{\colorbox{#1}{$\displaystyle #2$}}
%  convenient absolute value symbol
\newcommand{\abs}[1]{\vert #1 \vert}
%  convenient floor and ceiling
\newcommand{\floor}[1]{\lfloor #1 \rfloor}
\newcommand{\ceil}[1]{\lceil #1 \rceil}
%  convenient modulo
\newcommand{\Mod}[1]{\ \mathrm{mod}\ #1}
%  for logical not operator, iff symbol, convenient "if/then"
\renewcommand{\lnot}{\mathord{\sim}}
\let\iff\leftrightarrow
\let\Iff\Leftrightarrow
\let\then\rightarrow
\let\Then\Rightarrow
%  vectors
\newcommand{\vv}[1]{\boldsymbol{#1}}
\newcommand{\VV}[1]{\overrightarrow{#1}}
%  column vector
\newcommand{\cvv}[1]{\left(\begin{smallmatrix}#1\end{smallmatrix}\right)}
%  cell colours
\newcommand\bggreen{\cellcolor{green!10}}

\makeatother
\definecolor{myblue}{cmyk}{1,.72,0,.38}
\everymath\expandafter{\the\everymath \color{myblue}}
% Define BibTeX command
\def\BibTeX{{\rm B\kern-.05em{\sc i\kern-.025em b}\kern-.08em
    T\kern-.1667em\lower.7ex\hbox{E}\kern-.125emX}}

% Don't print section numbers
\setcounter{secnumdepth}{0}

\setlength{\parindent}{0pt}
\setlength{\parskip}{0pt plus 0.5ex}
%% this changes all items (enumerate and itemize)
\setlength{\leftmargini}{0.5cm}
\setlength{\leftmarginii}{0.5cm}
\setlist[itemize,1]{leftmargin=2mm,labelindent=1mm,labelsep=1mm}
\setlist[itemize,2]{leftmargin=2mm,labelindent=1mm,labelsep=1mm}

%My Environments
\newtheorem{example}[section]{Example}
% -----------------------------------------------------------------------

\begin{document}
\raggedright
\footnotesize


% multicol parameters
% These lengths are set only within the two main columns
\setlength{\columnseprule}{0.25pt}
\setlength{\premulticols}{1pt}
\setlength{\postmulticols}{1pt}
\setlength{\multicolsep}{1pt}
\setlength{\columnsep}{2pt}

\begin{center}
    \fbox{%
        \parbox{0.4\linewidth}{\centering \textcolor{black}{
            {\Large\textbf{MA1101R}}
            \\ \normalsize{AY20/21 sem 2}}
            \\ {\footnotesize \textcolor{myblue}{by jovyntls}}
        }%
    }
\end{center}


\begin{itemize}
    \item \textbf{Ex2.24} 
    \begin{itemize}
        \item (a) if $A$ and $B$ are diagonal matrices of the same size, then $AB = BA$
        \item (b) if $A$ is a square matrix, then $(A+A^T)$ is symmetric.
        \item (g) if $AA^T=0$, then $A=0$.
    \end{itemize}
    \item \textbf{Ex2.61} - if $A=PBP^{-1}$ then $\det(A)=\det(B)$.
    \item \textbf{Ex3.24} - if $V$ and $W$ are subspaces of $\mathbb{R}^n$,
    \begin{itemize}
        \item $V \cap W$ is a subspace of $\mathbb{R}^n$
        \item $V \cup W$ is a subspace of $\mathbb{R}^n \iff V \subseteq W$ or $W \subseteq V$
    \end{itemize}
    \item \textbf{Ex3.30} - let $u_1, u_2, \dots, u_k$ be vectors in $\mathbb{R}^n$ and $P$ be a square matrix of order $n$.
    \begin{itemize}
        \item if $Pu_1, Pu_2, \dots, Pu_k$ are linearly independent, then $u_1, u_2, \dots, u_k$ are linearly independent
        \item if $P$ is invertible and $u_1, u_2, \dots, u_k$ are linearly independent, then $Pu_1, Pu_2, \dots, Pu_k$ are linearly independent
        \item if $P$ is NOT invertible and $u_1, u_2, \dots, u_k$ are linearly independent, then $Pu_1, Pu_2, \dots, Pu_k$ are NOT necessarily linearly independent
    \end{itemize}
    \item \textbf{Ex4.10} - the linear relations between columns are not changed by row operations.
    \item \textbf{Ex4.22} - let $A$ be a $m \times n$ matrix and $P$ be a $m \times m$ matrix. if $P$ is invertible, $rank(PA) = rank(A)$
    \item \textbf{Ex4.25} - let $A$ be a $m \times n$ matrix. 
    \begin{itemize}
        \item The nullspace of $A$ is equal to the nullspace of $A^TA$.
        \item nullity($A$) = nullity($A^TA$)
        \item rank($A$) = rank($A^TA$)
    \end{itemize}
    \item \textbf{Ex5.32} - Let $A$ be an orthogonal matrix. $u, v$ are vectors in $\mathbb{R}^n$.
    \begin{itemize}
        \item $\| u \| = \| Au \|$
        \item $d(u, v) = d(Au, Av)$
        \item angle between $u$ and $v$ = angle between $Au$ and $Av$ 
    \end{itemize}
    \item \textbf{Ex5.32} - Let $A$ be an orthogonal matrix and $S = \{u_1, u_2, \dots, u_n\}$ be a basis for $\mathbb{R}^n$.
    \begin{itemize}
        \item $T = \{Au_1, Au_2, \dots, Au_n\}$ is a basis for $\mathbb{R}^n$.
        \item if $S$ is orthogonal, $T$ is orthogonal. 
        \item if $S$ is orthonormal, $T$ is orthonormal. 
    \end{itemize}
    \item \textbf{Ex6.23} - if $A$ is diagonalisable, $A^T$ is diagonalisable
    \item \textbf{Ex6.26} - if $A$ is symmetric and $u, v$ are 2 eigenvectors of $A$ associated with $\lambda$ and $\mu$, where $\lambda \neq \mu$, then $u \cdot v = 0$.
    \item \textbf{Ex7.10} - a linear operator $T$ is an isometry if $\|T(u)\| = \|u\|$ for all $u \in \mathbb{R}^n$.
    \begin{itemize}
        \item (a) $T(u) \cdot T(v) = u \cdot v$ for all $u, v \in \mathbb{R}^n$
        \item (b) $T$ is an isometry $\iff$ the standard matrix is an orthogonal matrix
        \item (c) all isometries on $\mathbb{R}^n$ are of the form
            \\* $T(\begin{pmatrix}
                x\\y
            \end{pmatrix}) = \begin{pmatrix}
                x\cos\theta + \delta y\sin\theta \\
                y\sin\theta - \delta y\cos\theta
            \end{pmatrix}$ for $\begin{pmatrix}
                x\\y
            \end{pmatrix} \in \mathbb{R}^2$
            where $\delta = \pm 1$ and $0 \leq \theta < 2\pi$
    \end{itemize}
    \item \textbf{LAB4} - if $AA^T$ is a diagonal matrix, then the rows of $A$ form an orthogonal set.
    \item \textbf{LAB4} - if $AA^T$ is an identity matrix, then the rows of $A$ form an orthonormal set.
\end{itemize}

\begin{itemize}
    \item to show $A$ is invertible: show $\exists B$ s.t. $AB = I$ and $BA=I$
\end{itemize}




\end{document}