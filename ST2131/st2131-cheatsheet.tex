\documentclass[10pt, landscape]{article}
\usepackage[scaled=0.92]{helvet}
\usepackage{calc}
\usepackage{multicol}
\usepackage[a4paper,margin=3mm,landscape]{geometry}
\usepackage{amsmath,amsfonts,amssymb}
\usepackage{color,graphicx,overpic}
\usepackage{hyperref}
\usepackage{newtxtext} 
\usepackage{enumitem}
\usepackage[table]{xcolor}
\usepackage{mathtools}
\setlist{nosep}
% for including images
\graphicspath{ {./images/} }

\pdfinfo{
  /Title (ST2131.pdf)
  /Creator (TeX)
  /Producer (pdfTeX 1.40.0)
  /Author (Jovyn Tan)
  /Subject (ST2131)
/Keywords (ST2131, nus,cheatsheet,pdf)}

% Turn off header and footer
\pagestyle{empty}

% redefine section commands to use less space
\makeatletter
\renewcommand{\section}{\@startsection{section}{1}{0mm}%
  {-1ex plus -.5ex minus -.2ex}%
  {0.5ex plus .2ex}%x
{\normalfont\large\bfseries}}
\renewcommand{\subsection}{\@startsection{subsection}{2}{0mm}%
  {-1explus -.5ex minus -.2ex}%
  {0.5ex plus .2ex}%
{\normalfont\normalsize\bfseries}}
\renewcommand{\subsubsection}{\@startsection{subsubsection}{3}{0mm}%
  {-1ex plus -.5ex minus -.2ex}%
  {1ex plus .2ex}%
{\normalfont\small\bfseries}}%
\makeatother

\renewcommand{\familydefault}{\sfdefault}
\renewcommand\rmdefault{\sfdefault}
%  makes nested numbering (e.g. 1.1.1, 1.1.2, etc)
\renewcommand{\labelenumii}{\theenumii}
\renewcommand{\theenumii}{\theenumi.\arabic{enumii}.}
\renewcommand\labelitemii{•}
\renewcommand\labelitemiii{•}

\definecolor{mathblue}{cmyk}{1,.72,0,.38}
\everymath\expandafter{\the\everymath \color{mathblue}}

% Don't print section numbers
\setcounter{secnumdepth}{0}

\setlength{\parindent}{0pt}
\setlength{\parskip}{0pt plus 0.5ex}
%% adjust spacing for all itemize/enumerate
\setlength{\leftmargini}{0.5cm}
\setlength{\leftmarginii}{0.5cm}
\setlist[itemize,1]{leftmargin=2mm,labelindent=1mm,labelsep=1mm}
\setlist[itemize,2]{leftmargin=4mm,labelindent=1mm,labelsep=1mm}

% adding my commands
% tightcenter
\newenvironment{tightcenter}{%
  \setlength\topsep{0pt}
  \setlength\parskip{0pt}
  \begin{center}
    }{%
  \end{center}
}

% boxed
\newenvironment{tightbox}{%
  \setlength\topsep{0pt}
  \setlength\parskip{0pt}
  \begin{center}
    \begin{tabular}{|@{\hspace{\dimexpr\fboxsep+0.5\arrayrulewidth}}c@{\hspace{\dimexpr\fboxsep+0.5\arrayrulewidth}}|}
      \hline
    }
    {%
    \\ \hline
    \end{tabular}
  \end{center}
}

% fixed width box
\newenvironment{fixedbox}[1][0.7]{
  \setlength\topsep{0pt}
  \setlength\parskip{0pt}
  \begin{center}
    \begin{tabular}{|>{\centering\arraybackslash}m{#1\linewidth}|}
    \hline
  }{
  \\ \hline
  \end{tabular}
  \end{center}
}

% definition of a new term
\usepackage{soul}
\definecolor{paleyellow}{RGB}{251,243,218}
\newcommand{\definition}[2][]{\sethlcolor{paleyellow}\hl{\textbf{#2}} #1  $\rightarrow$}

% important note (attention)
\newcommand{\attention}{{\color{red}\textbf{! }}}


% proofs
\newenvironment{proof}
{%
  \sbox0{\textit{Proof}: }%
  \list{}{\labelwidth\wd0 \leftmargin\wd0 \labelsep 0pt }
\item[\usebox0]}
  {\endlist}


% -----------------------------------------------------------------------

\begin{document}
\raggedright
\footnotesize
\begin{multicols*}{3}
  % multicol parameters
  \setlength{\columnseprule}{0.25pt}

  \begin{center}
    \fbox{%
      \parbox{0.8\linewidth}{\centering \textcolor{black}{
          {\Large\textbf{ST2131}}
        \\ \normalsize{AY21/22 SEM 2}}
        \\ {\footnotesize \textcolor{gray}{github/jovyntls}}
      }%
    }
  \end{center}

  \section{01. COMBINATORIAL ANALYSIS}

  \textbf{tricky} - E18, E20-22, E23, E26

  \subsection{The Basic Principle of Counting}

  \begin{itemize}
    \item \definition{combinatorial analysis} the mathematical theory of counting
    \item \definition{basic principle of counting} Suppose that two experiments are performed. 
      If exp1 can result in any one of $m$ possible outcomes and if, for each outcome of exp1, there are $n$ possible outcomes of exp2, 
      then together there are $mn$ possible outcomes of the two experiments.
    \item \definition{generalized basic principle of counting} If $r$ experiments are performed such that the first one may result in any of $n_1$ possible outcomes and if for each of these $n_1$ possible outcomes, and if ..., then there is a total of $n_1 \cdot n_2 \cdot \dots \cdot n_r$ possible outcomes of $r$ experiments.
  \end{itemize}

  \subsection{Permutations}

  \textbf{factorials} - $1! = 0! = 1$

  \textbf{N1} - if we know how to count the number of different ways that an event can occur, we will know the probability of the event.

  \textbf{N2} - there are $n!$ different arrangements for $n$ objects.

  \textbf{N3} - there are $\frac{n!}{n_1!\, n_2!\, \dots n_r!}$ different arrangements of $n$ objects, 
  of which $n_1$ are alike, $n_2$ are alike, ..., $n_r$ are alike.

  \subsection{Combinations}

  \textbf{N4} - $\binom{n}{r} = \frac{n!}{(n-r)!\,r!}$ represents the number of different groups of size $r$ that could be selected from a set of $n$ objects when the order of selection is not considered relevant.

  \textbf{N4b} - $\binom{n}{r} = \binom{n-1}{r-1} + \binom{n-1}{r}, \quad 1 \leq r \leq n$
  \begin{proof}
    If object 1 is chosen $\Rightarrow$ $\binom{n-1}{r-1}$ ways of choosing the remaining objects.
    \\* If object 1 is not chosen $\Rightarrow$ $\binom{n-1}{r}$ ways of choosing the remaining objects.
  \end{proof}

  \textbf{N5} - \textbf{The Binomial Theorem} - \( {\displaystyle{(x+y)^n = \sum^n_{k=0} \binom{n}{k} x^k y^{n-k} }} \) 
  \begin{proof}
    by mathematical induction: $n=1$ is true; expand; sub dummy variable; combine using N4b; combine back to final term
  \end{proof}

  \subsection{Multinomial Coefficients}

  \textbf{N6} - $\binom{n}{n_1, n_2, \dots, n_r} = \frac{n!}{n_1!\, n_2!\, \dots \, n_r!}$  
  represents the number of possible divisions of $n$ distrinct objects 
  into $r$ distinct groups of respective sizes $n_1, n_2, \dots, n_3$, 
  where $n_1 + n_2 + \dots + n_r = n$
  \begin{proof}
    using basic counting principle, 
    \\* $= \binom{n}{n_1} \binom{n-n_1}{n_2} \binom{n-n_1-n_2}{n_3} \dots \binom{n-n_1-n_2-n_{r-1}}{n_r}$
    \\* $= \frac{n!}{(n-n_1)!\,n_1!} \times \frac{(n-n_1)!}{(n-n_1-n_2)!\, n_2!} \times \dots \times \frac{( n-n_1-n_2-\dots -n_{r-1} )}{0!\, n_r!}$
    \\* $= \frac{n!}{n_1!\, n_2! \, \dots \, n_r!}$
  \end{proof}

  \textbf{N7} - \textbf{The Multinomial Theorem}: $(x_1 + x_2 + \dots + x_r)^n$
  \\* $ \quad\quad\quad = \sum\limits_{(n_1, \dots, n_r):n_1 + n_2 + \dots + n_r = n} \frac{n!}{n_1! \, n_2! \, \dots n_r!} x_1^{n_1} x_2^{n_2} \dots x_r^{n_r}$

  \subsection{Number of Integer Solutions of Equations}

  \textbf{N8} - there are $\binom{n-1}{r-1}$ distinct \textit{positive} integer-valued vectors $(x_1, x_2, \dots, x_r)$ satisfying
  $x_1 + x_2 + \dots + x_r = n, \quad x_i > 0, \quad i = 1,2,\dots,r$  
  \\* \attention cannot be directly applied to \textit{N8} as 0 value is not included

  \textbf{N9} - there are $\binom{n+r-1}{r-1}$ distinct  \textit{non-negative} integer-valued vectors $(x_1, x_2, \dots, x_r)$ satisfying $x_1 + x_2 + \dots + x_r = n$ 
  \begin{proof}
    let $y_k = x_k + 1 \Rightarrow y_1 + y_2 + \dots + y_r = n + r$
  \end{proof}








































\end{multicols*}

\end{document}
