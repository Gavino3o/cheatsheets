%%%%%%%%%%%%%%%%%%%%%%%%%%%%%%%%%%%%%%%%%%%%%%%%%%%%%%%%%%%%%%%%%%%%%%
% writeLaTeX Example: A quick guide to LaTeX
%
% Source: Dave Richeson (divisbyzero.com), Dickinson College
% 
% A one-size-fits-all LaTeX cheat sheet. Kept to two pages, so it 
% can be printed (double-sided) on one piece of paper
% 
% Feel free to distribute this example, but please keep the referral
% to divisbyzero.com
% 
%%%%%%%%%%%%%%%%%%%%%%%%%%%%%%%%%%%%%%%%%%%%%%%%%%%%%%%%%%%%%%%%%%%%%%
% How to use writeLaTeX: 
%
% You edit the source code here on the left, and the preview on the
% right shows you the result within a few seconds.
%
% Bookmark this page and share the URL with your co-authors. They can
% edit at the same time!
%
% You can upload figures, bibliographies, custom classes and
% styles using the files menu.
%
% If you're new to LaTeX, the wikibook is a great place to start:
% http://en.wikibooks.org/wiki/LaTeX
%
%%%%%%%%%%%%%%%%%%%%%%%%%%%%%%%%%%%%%%%%%%%%%%%%%%%%%%%%%%%%%%%%%%%%%%

\documentclass[10pt,landscape]{article}
\usepackage{amssymb,amsmath,amsthm,amsfonts}
\usepackage{multicol,multirow}
\usepackage{calc}
\usepackage{ifthen}
\usepackage{graphicx}
\usepackage{listings}
\usepackage{fancyvrb}
\usepackage[landscape]{geometry}
\usepackage[colorlinks=true,citecolor=blue,linkcolor=blue]{hyperref}


\ifthenelse{\lengthtest { \paperwidth = 11in}}
    { \geometry{top=.3in,left=.3in,right=.3in,bottom=.3in} }
  {\ifthenelse{ \lengthtest{ \paperwidth = 297mm}}
    {\geometry{top=1cm,left=1cm,right=1cm,bottom=1cm} }
    {\geometry{top=1cm,left=1cm,right=1cm,bottom=1cm} }
  }
\pagestyle{empty}
\makeatletter
\renewcommand{\section}{\@startsection{section}{1}{0mm}%
                                {-1ex plus -.5ex minus -.2ex}%
                                {0.5ex plus .2ex}%x 
                                {\normalfont\normalsize\bfseries}}
\renewcommand{\subsection}{\@startsection{subsection}{2}{0mm}%
                                {-1explus -.5ex minus -2ex}%
                                {0.2ex plus -1ex}%
                                {\normalfont\small\bfseries}}
\renewcommand{\subsubsection}{\@startsection{subsubsection}{3}{0mm}%
                                {-1ex plus -.5ex minus -.5ex}%
                                {0.2ex plus -1ex}%
                                {\normalfont\footnotesize\bfseries}}
\makeatother
\setcounter{secnumdepth}{0}
\setlength{\parindent}{0pt}
\setlength{\parskip}{0pt plus 0.5ex}
\graphicspath{ {./images/} }
% -----------------------------------------------------------------------

\title{ACC1701X Midterm Cheatsheet AY23/24 Sem 2}

\begin{document}

\raggedright
\footnotesize

\begin{center}
     \Large{\textbf{ACC1701X Midterm Cheatsheet AY23/24 Sem 2}} \\
\end{center}
\begin{multicols}{3}
\setlength{\premulticols}{1pt}
\setlength{\postmulticols}{1pt}
\setlength{\multicolsep}{1pt}
\setlength{\columnsep}{2pt}
\begin{scriptsize}

\subsection{Financial Accounting}
\textbf{Financial Accounting:} provides financial information primarily for external users to make decisions.\\
\textbf{Management Accounting:} provides financial information primarily for external users to make decisions.
\begin{itemize} \itemsep -2pt
    \item Provides information to management and staff for planning, implementation and control purposes to improve business performance.
    \item Provides reports which include budgets, cost analysis and 
    divisional performance reports.
    \item Primarily for internal users, i.e., management and staff.
\end{itemize}
\textbf{International Financial Reporting Standards (IFRS)} set by IASB (International Accounting Standard Board) is primarily used for this course.

\subsubsection*{Financial Accounting Concepts and Assumptions}
\begin{itemize} \itemsep -2pt
    \item \textbf{Separate Entity Concept:} activities of a business entity are separated from those of the individual owners.
    \item \textbf{Time-period assumption:} Long life of a company can be reported in a shorter time period (annually [fiscal year], quarterly, monthly)
    \item \textbf{Arm's-length transaction assumption:} business dealings between entities are conducted in a rational basis and all parties are acting for their own interests.
    \item \textbf{Cost principle:} financial statement items are measured at their historical costs/original costs at the transaction date.
    The historical cost is assumed to represent the fair market value of the item at the date of transaction.
    \item \textbf{Fair value principle:} the assets and liabilities should be 
    measured at fair value to improve relevance of accounting information.
    \item \textbf{Monetary measurement concept:} Items in financial statements must have value measurable in dollar value.
    \item \textbf{Going concern assumption:} When preparing the financial 
    statements, the business entity is assumed to be able to sustain itself for the foreseeable future (not under liquidation)
\end{itemize}

\subsubsection*{Limitations of Balance Sheet}
\begin{itemize} \itemsep -2pt
    \item Assets are recorded at their purchase cost, not their current market value.
    \item Not all economic assets are included in the balance sheet. 
    (e.g., internally generated brand, good reputation and human talents)
    \item The accounting book value of a company (measured by the amount of equity) is usually less 
    than the company’s market value (measured by market price per share $\times$ number of shares).
\end{itemize}

\subsubsection*{Financial Statements}
\begin{itemize} \itemsep -2pt
    \item Statement of Comprehensive Income (SCI)
    \item Balance Sheet (Statement of Financial Position)\\
    \item Statement of Changes of equity
    \item Statement of Cash Flows
    \item Notes to Financial Statements
\end{itemize}

\subsubsection*{Balance Sheet Items}
\textbf{Assets:} Cash, Accounts Receivable, Inventory, Buildings, Property, Plant and Equipment [Liquidity]\\
\textbf{Liabilities:} Accounts Payable, Income Taxes Payable, Mortgage Payable, Unearned Revenue [Due date]\\
\textbf{Equity:} Capital Stock, Retained Earnings [Dividends distributed reduce retained earnings]\\
\begin{itemize} \itemsep -2pt
    \item $Assets = Liabilities + Equity$
    \item $Assets = \textit{Current Asset} + \textit{Non-Current Assets}$
    \item $Liabilities = \textit{Current Liabilities} + \textit{Non-Current Liabilities}$
    \item $\textit{Retained Earnings(begin)} + \textit{Net Income} - Dividends = \textit{Retained Earnings(end)}$
    \item $\textit{Equity(begin)} + \textit{Increase in Capital Stock} + \textit{Net Income} - Dividends + \textit{Other Comprehensive Income} = Equity(end)$
\end{itemize}

\subsubsection*{SCI Items}
\begin{itemize} \itemsep -2pt
    \item \textit{Expenses = Operating expense $+$ Non-Operating Expense}
    \item $Revenue - Expenses = \textit{Net Income}$\\
    \item $\textit{Net Income} + \textit{Other Comprehensive Income} = \textit{Comprehensive Income}$\\
    \item $\textit{Revenue} - \textit{COGS} = \textit{Gross Profit}$
\end{itemize}

\subsubsection*{Notes to Financial Statements}
\begin{itemize} \itemsep -2pt
    \item Summary of Significant Accounting Policies
    \item Additional Information about the Summary Totals 
    \item Disclosure of Information Not Recognized in the financial statements
    \item Supplementary Information required under the accounting standards
\end{itemize}

\subsection*{Mechanics of Accounting Cycle}
1. Analyse transactions\\
2. Record effects of transaction\\
3. Summarise effects of transactions (journal entries \& trial balance)\\
4. Prepare reports (adjusting entries, financial statements, closing the books)\\

\textbf{Flow of Accounting Records:}\\
Journal Entries, T Accounts, Trial Balance, Income Statement, Statement of Changes in Equity, Balance Sheet\\

\textbf{Criteria for transaction recognition:}\\
1. Involve exchange of resources\\
2. Conducted at Arm's Length between two independent entities\\
3. Can be reliably measured\\

Business Docs such as sales invoices, official receipts, purchases orders, suppliers' invoices capture
business transaction. They provide independent objective evidence of transaction, establish dollar amount, and
facilitate analysis of business events.\\

\subsubsection*{Journal Entry}
A recording of a business transactions; usually includes 
a debit entry and a credit entry to the relevant 
accounts with amount, a date and sometimes an 
explanation of the transaction.\\

\underline{\textbf{Asset Accounts:}}\\
Typically has debit balances.\\DR Asset (Increase) CR Asset (Decrease)\\
\underline{\textbf{Liability or Equity Accounts:}}\\
Typically has credit balances.\\DR Liability/Equity (Decrease) CR Liability/Equity (Increase)\\
\underline{\textbf{Revenue and Expenses Accounts:}}\\
Expenses typically has debit balances. \\Revenue typically has credit balances\\
DR Expense (Increase) CR Expense (Decrease)\\
DR Revenue (Decrease) CR Revenue (Increase)\\
\underline{\textbf{Capital Stock, Retained Earnings, Dividends}}\\
DR Capital Stock (Decrease) CR Capital Stock (Increase)\\
DR Retained Earnings (Decrease) CR Retained Earnings (Increase)\\
DR Dividends (Increase) CR Dividends (Decrease)\\
\underline{\textbf{Personal journal entries notes:}}\\
- Try to think in terms of cash flow\\
- DR Cash -- CR Revenue\\
- DR Expense -- CR Cash\\
- DR Cash -- CR Capital Stock\\
- DR Cash -- CR Revenue; DR COGS -- CR Inventory\\
- DR Notes Payable, Interest Expense -- CR Cash (paying back notes)\\
- DR Dividends -- CR Cash\\
\subsubsection{T-Accounts}
- Basically combine all journal entry transactions by accounts used and calculate ending balances (can be debit or credit).\\
\subsubsection*{Trial Balance}
- Prepare trial balance using the ending balances from \underline{all} T-accounts.\\
- Total debit balance should be \underline{equivalent} to total credit balance.

\subsection*{Accrual Accounting}
System of accounting in which 
revenues are recognized when certain criteria are satisfied 
(i.e., revenues are earned); and expenses are recorded as 
they are incurred to generate the revenue (matching 
principle). This is regardless of \underline{when} cash is paid or 
received.\\
- Revenues can be recognized \textbf{even cash is not received}.\\
- Expenses can be recorded \textbf{even cash is not paid}.\\
- The matching principle requires that all costs and expenses 
incurred in generating revenues must be recognized in the 
same reporting period as the related revenues.\\
- \textbf{Follows the time-period assumption}
which is that the life of a business is divided into distinct 
and relatively short time periods(e.g., 12-month) so that 
accounting information can be timely. 

\subsubsection*{Adjusting entries}
Prepared at the end of each accounting period to adjust accounts to their proper amount on an accrual basis.\\
Do not involve cash. Involves a BL account or SCI account.

\subsubsection*{\underline{1. Accrued/Unrecorded receivable (asset)}}
In Arrears. Revenues earned during a period that have not been 
recorded by the end of that period. Represent amount of cash or resources to be 
collected by the entity in the future. \\
\textbf{DR Accounts Receivable -- CR Revenue}

\subsubsection*{\underline{2. Accrued/Unrecorded liabilities (liability)}}
Expenses incurred and not paid during a period and are not
recorded at the end of that period. Represents entity's obligation to pay for 
the expense in the future.\\
\textbf{DR Expense -- CR Payable}\\
Example: DR Interest Expense -- CR Interest Payable (Adj Entry)\\
Dr Interest Payable, Loan Payable -- CR Cash 

\subsubsection*{\underline{3. Prepaid Expense (asset)}}
Payments made in advance for expense items, e.g., prepaid 
insurance or rent.\\
Journal Entry (day of payment):\\
\textbf{DR Prepaid Expense -- CR Cash}\\
Adjusting Entry (expense \underline{incurred up to date})\\
\textbf{DR Expense -- CR Prepaid Expense}\\
- Calculate fraction of prepaid expense used up at current date\\
- Supplies Example: DR Supplies -- CR Cash; DR Supplies Expense -- CR Supplies;

\subsubsection*{\underline{4. Unearned Revenue (liability)}}
Cash amounts received from customers before its 
corresponding revenue can be recognized.\\
Journal Entry (Customer advance payment):\\
\textbf{DR Cash -- CR Unearned Revenue}\\
Adjusting Entry (revenue \underline{earned up to date}):\\
\textbf{DR Unearned Revenue -- CR Revenue}
- Calculate fraction of revenue earned using time\\

\subsubsection*{\underline{Closing the books}}
Adjust trial balance $\rightarrow$ Prepare SCI $\rightarrow$ Ending Retained Earnings $\rightarrow$ Closing books\\
Making all the Revenue, Expense, Retained Earnings and Dividends account balance to zero.\\
Debit and Credit balances are swapped for these entries.
\begin{itemize} \itemsep -2pt
    \item \textbf{DR Revenue -- CR Expense, Retained Earnings (Net income)}
    \item \textbf{DR Retained Earnings -- CR Dividends}
\end{itemize}

\subsection*{Revenue and Receivables}
\subsubsection*{Revenue Recognition}
1. Identify the contract with the customer.\\
2. Identify performance obligation(s) in the contract.\\
3. Determine the transaction price.\\
4. Allocate the transaction price to the separate performance 
obligations (if more than one) [pro-rate].\\
5. Determine when the performance obligation is satisfied 
and revenue can be recognized.\\

\subsubsection*{Sales Discount}
To encourage customers to pay early, companies use sales 
discounts (also called cash discount) as an early payment 
incentive.\\
\begin{itemize} \itemsep -2pt
    \item \textbf{credit terms of 2/10, n/30} means that a sales discount has been given.
    \item Customer deducts 2\% from invoice price if paid within 10 days.
    Customer has to pay full sales price within maximum of 30 days.
    \item \textbf{DR Accounts Receivable -- CR Sales Revenue} (Revenue made, credit payment)
    \item \textbf{DR Cash, Sales Discount -- Accounts Receivable}
\end{itemize}

\subsubsection*{Sales Returns and Allowances}
\textbf{JE: DR Accounts Receivable -- CR Sales Revenue}\\
\textbf{JE: DR COGS -- Inventory}\\
If return required:\\
\textbf{JE: DR Sales Returns and Allowances -- CR Accounts Receivable}\\
\textbf{JE: DR Inventory -- CR COGS}\\

\textit{Sales Revenue $-$ Sales Discount $-$ Sales Returns and Allowances $=$ Net Sales}

\subsubsection*{Treatment of Bad Debt}
- Uncollectible accounts receivable\\
- Allowance method estimates and records the amount of 
impairment of accounts receivable before they become 
uncollectible.\\
-  Direct write-off method is NOT allowed under IFRS as 
it violates the matching principle.\\
ECL is an \underline{estimated} expense in the SCI. Loss Allowance is a contra-asset account to Accounts Receivable in the balance sheet.\\
\textbf{DR Expected Credit Loss -- CR Loss Allowance}\\
If specific customer identified specifically as uncollectible, \underline{write off}:\\
\textbf{DR Loss Allowance -- CR Accounts Receivable}\\
If written off subsequently pays the outstanding balance, \underline{reinstate}:\\
\textbf{DR Accounts Receivable -- CR Loss Allowance}\\
\textbf{DR Cash -- CR Accounts Receivable}\\

\textit{Accounts Receivable(net) = Accounts Receivable(begin) $-$ Loss Allowance (end)}\\

\textbf{Estimating Loss Allowance (end)}\\
Estimate the amount of uncollectible as a percentage of 
the total receivables balance at the end of the period.\\
Normally has credit balance.\\
1. \textbf{Percentage of Total Receivables:}\\
$\text{ Total Accounts Receivable} \times \text{Percentage Uncollectible(\%)}$\\
2. \textbf{Aging Method:}\\
Each receivable is categorized by the number of days it has 
been outstanding.\\
Multiplied by an appropriate 
uncollectible percentage. The older the 
receivable, the less likely the company is to collect.\\
$\sum \text{Accounts Receivables in Age Category} \times \text{Respective Percentage Uncollectible(\%)}$\\
- If there is an existing \textbf{credit} balance in loss allowance. Top up less by $ECL - \textit{Unadjusted Balance}$\\
- If there is an existing \textbf{debit} balance in loss allowance. Top up more by $ECL + \textit{Unadjusted Balance}$\\


\subsubsection*{Notes Receivable}
A legal contract in the form of a promissory note 
received by a company and written by its customer to settle 
accounts receivable after the credit period. A company can also lend 
money to an external party by accepting a promissory note.\\
\textit{Interest Revenue = Face Value $\times$ Annual Interest Rate $\times$ Term of the node(credit duration)}\\
\begin{itemize} \itemsep -2pt
    \item \textbf{DR Notes Receivable -- CR Accounts Receivable} (issuing note)
    \item \textbf{DR Cash -- CR Notes Receivable, Interest Revenue} (accepting payment)
\end{itemize}
If note is \underline{dishonoured} (convert into AR):
\begin{itemize} \itemsep -2pt
    \item \textbf{DR Accounts Receivable -- CR Notes Receivable, Interest Revenue}
\end{itemize}

\subsubsection*{Foreign Currency}
Reported in SCI as \textbf{Foreign exchange loss/gain}.\\
To reduce: Denominate the transaction in U.S. dollars; or
Enter into a forward contract to hedge the exchange risk
\includegraphics*[width=0.6\linewidth]{foreign_currency.png}

\subsubsection*{Efficiency Ratios}
\begin{itemize} \itemsep -1pt
    \item \textit{Average AR = (AR(begin) + AR(end))/2}
    \item $\text{Accounts Receivable Turnover} = \frac{\text{Net Sales}}{\text{Average AR}}$
    \item  $\text{Average Collection Period} = \frac{365}{\text{Accounts Receivable Turnover}}$
\end{itemize}
The higher AR Turnover, the better.\\
The shorter Average Collection Period, the better\\
Not \textbf{Cash Poor}

\subsection*{Cash and Internal Controls}
\subsubsection*{Purchase Discounts}
\textbf{DR Inventory -- CR Accounts Payable}\\
\textbf{DR Accounts Payable -- CR Cash, Inventory}(discount amount on remaining AP)\\
Related to cost-principle.

\subsubsection*{Debit Memo}
Returning stuff as a buyer.\\
\textbf{DR Accounts Payable -- CR Inventory}\\

\subsubsection*{Bank Reconcilation}
\includegraphics*[width=0.6\linewidth]{bank_reconcilation.png}\\
For bank errors, the company will notify the bank and have 
the bank make the corrections. No journal entries required\\
\underline{\textbf{NSF (Not Sufficient Funds)}}\\
Adjusting entry: \textbf{DR Accounts Receivable -- CR Cash}
\underline{\textbf{Misc Items unrecorded in books}}\\
Adjusting entry: \textbf{DR Service Expense -- CR Cash}\\
Adjusting entry: \textbf{DR Cash -- CR Interest Revenue}\\
\underline{\textbf{Bank Transfers}}\\
Adjusting entry: \textbf{DR Expenses -- CR Cash}\\
\underline{\textbf{Direct Deposits}}\\
Adjusting entry: \textbf{DR Cash -- CR Accounts Receivable}\\

\subsubsection*{Petty Cash Fund and Vouchers}
Setting up:\\
\textbf{DR Petty Cash -- CR Cash}\\
Replenishing:\\
\textbf{DR Various Expenses..., Cash Short and Over -- CR Cash(Initial petty - remaining)}

\subsubsection*{Internal Controls}
\includegraphics*[width=0.8\linewidth]{internal_controls.png}\\

\subsubsection*{Fraud}
\begin{itemize} \itemsep -2pt
    \item Asset misappropriation
    \item Corruption (bribery and kickbacks)
    \item Financial misstatement or omission
\end{itemize}

\subsubsection*{Earning Management}
Intentional, to manipulate financial results legally. To meet internal targets, external expectations, smooth income 
and window dress for public stock offering or loan application.\\
Two common ways:\\
- modify operating or investing decisions. E.g., changing the timing 
to invest or purchase goods and delaying expenditure;\\
- modify accounting estimates and methods. E.g., changing 
accounting estimate of loss allowance for accounts receivable.\\

\subsection*{Inventory and Cost of Sales}
The cost of inventories shall comprise all costs of purchase and other 
costs incurred in bringing the inventories to their present location and 
condition. (costs directly attributable to the acquisition of goods.)\\
Excludes sales and marketing cost, those are operating expenses.\\
- FOB Destination (Ownership passes to buyer at destination)\\
- FOB Shipping Point (Ownership passes to buyer at shipping point)\\
There is no passing of ownership between the consignee and 
consignor (third party). Goods still belong to consignor.\\

\subsubsection*{Perpetual vs Periodic Method}
\textbf{Perpetual:} Inventory and COGS are recorded on a continual basis for 
each purchase and sales transaction.\\
\textbf{Periodic:}  Amounts of COGS and Inventory reported at year end, are 
computed based on the physical manual stock-take at 
year-end. Temporary accounts are used to record the \textbf{purchases, 
freight-in, purchases returns and purchases discounts}.\\
Freight-in is the cost of having goods or materials delivered to a business for manufacture or resale.\\
Basically, replace all "Inventory" with temporary detailed accounts.\\
\textbf{\underline{NO} DR COGS -- CR Inventory is recorded during date of sale for Periodic System.}

\subsubsection*{Periodic System}
At the end of reporting period (year), two steps:\\
1. All temporary accounts (purchases, freight-in, purchases returns and purchases discounts) are
adjusted to Inventory account.\\
\textbf{DR: Inventory(Net Purchases), Purchase Returns, Purchase Discounts -- CR: Freight In, Purchases}\\
2. Conduct Stock count or stock take at year-end to determine inventory(end) and compute COGS.\\
\textit{COGS $=$ Inventory(begin) $+$ Net Purchases $-$ Inventory(end)}\\
Then do one journal entry:\\
\textbf{DR COGS -- CR Inventory}

\subsubsection*{Shrinkage:}
Physical counts allow companies to determine inventory 
shrinkage. The difference between, the physical count’s 
value and the book’s value will be reported as the physical 
stock shrinkage. Reported as COGS.

\vspace*{-2pt}

\subsubsection*{Inventory Costing}
\begin{itemize} \itemsep -2pt
    \item Specific Identification 
    \item FIFO (End Inv count from most recent)
    \item LIFO (Not allowed under IFRS)
    \item Weighted Average Cost
\end{itemize}
\textit{Cost of goods available for sale $=$ Beginning Inv Balance + Net Purchases}\\
LIFO reports the Cost of goods with the most recent costs.  FIFO reports inventory with the most recent cost.\\
When inventory price rises. Gross margin, net income, inventory(end) is highest with FIFO. WA and LIFO understate inventory(end).

\subsubsection*{Net Realizable Value}
If NRV $<$ Cost, write down item by item:\\
\textbf{DR COGS -- CR Allowance for Inventory Write Down (Contra-Inventory, BL)}\\
If NRV $\geq$ Cost, no adjustments needed.\\
\underline{NOT ALLOWED} to write inventories down on the basis of a classification 
of inventory.

\subsubsection*{Efficiency Ratios}
- $\text{Inventory Turnover} = \frac{\text{COGS}}{\text{Average Inventory}}$ (higher, better)\\
- $\text{Num of Days Sales in Inventory} = \frac{365}{\text{Inventory Turnover}}$ (shorter, better)\\

\end{scriptsize}

\end{multicols}

\end{document}