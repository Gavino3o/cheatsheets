%%%%%%%%%%%%%%%%%%%%%%%%%%%%%%%%%%%%%%%%%%%%%%%%%%%%%%%%%%%%%%%%%%%%%%
% writeLaTeX Example: A quick guide to LaTeX
%
% Source: Dave Richeson (divisbyzero.com), Dickinson College
% 
% A one-size-fits-all LaTeX cheat sheet. Kept to two pages, so it 
% can be printed (double-sided) on one piece of paper
% 
% Feel free to distribute this example, but please keep the referral
% to divisbyzero.com
% 
%%%%%%%%%%%%%%%%%%%%%%%%%%%%%%%%%%%%%%%%%%%%%%%%%%%%%%%%%%%%%%%%%%%%%%
% How to use writeLaTeX: 
%
% You edit the source code here on the left, and the preview on the
% right shows you the result within a few seconds.
%
% Bookmark this page and share the URL with your co-authors. They can
% edit at the same time!
%
% You can upload figures, bibliographies, custom classes and
% styles using the files menu.
%
% If you're new to LaTeX, the wikibook is a great place to start:
% http://en.wikibooks.org/wiki/LaTeX
%
%%%%%%%%%%%%%%%%%%%%%%%%%%%%%%%%%%%%%%%%%%%%%%%%%%%%%%%%%%%%%%%%%%%%%%

\documentclass[10pt,landscape]{article}
\usepackage{amssymb,amsmath,amsthm,amsfonts}
\usepackage{multicol,multirow}
\usepackage{calc}
\usepackage{ifthen}
\usepackage{graphicx}
\usepackage{listings}
\usepackage{fancyvrb}
\usepackage[landscape]{geometry}
\usepackage[colorlinks=true,citecolor=blue,linkcolor=blue]{hyperref}


\ifthenelse{\lengthtest { \paperwidth = 11in}}
    { \geometry{top=.3in,left=.3in,right=.3in,bottom=.3in} }
  {\ifthenelse{ \lengthtest{ \paperwidth = 297mm}}
    {\geometry{top=1cm,left=1cm,right=1cm,bottom=1cm} }
    {\geometry{top=1cm,left=1cm,right=1cm,bottom=1cm} }
  }
\pagestyle{empty}
\makeatletter
\renewcommand{\section}{\@startsection{section}{1}{0mm}%
                                {-1ex plus -.5ex minus -.2ex}%
                                {0.5ex plus .2ex}%x 
                                {\normalfont\normalsize\bfseries}}
\renewcommand{\subsection}{\@startsection{subsection}{2}{0mm}%
                                {-1explus -.5ex minus -2ex}%
                                {0.2ex plus -1ex}%
                                {\normalfont\small\bfseries}}
\renewcommand{\subsubsection}{\@startsection{subsubsection}{3}{0mm}%
                                {-1ex plus -.5ex minus -.5ex}%
                                {0.2ex plus -1ex}%
                                {\normalfont\footnotesize\bfseries}}
\makeatother
\setcounter{secnumdepth}{0}
\setlength{\parindent}{0pt}
\setlength{\parskip}{0pt plus 0.5ex}
\graphicspath{ {./images/} }
% -----------------------------------------------------------------------

\title{ACC1701X Midterm Cheatsheet AY23/24 Sem 2}

\begin{document}

\raggedright
\footnotesize

\begin{center}
     \Large{\textbf{ACC1701X Midterm Cheatsheet AY23/24 Sem 2}} \\
\end{center}
\begin{multicols}{3}
\setlength{\premulticols}{1pt}
\setlength{\postmulticols}{1pt}
\setlength{\multicolsep}{1pt}
\setlength{\columnsep}{2pt}
\begin{scriptsize}

\subsection{Financial Accounting}
\textbf{Financial Accounting:} provides financial information primarily for external users to make decisions.\\
\textbf{Management Accounting:} provides financial information primarily for external users to make decisions.
\begin{itemize} \itemsep -2pt
    \item Provides information to management and staff for planning, implementation and control purposes to improve business performance.
    \item Provides reports which include budgets, cost analysis and 
    divisional performance reports.
    \item Primarily for internal users, i.e., management and staff.
\end{itemize}
\textbf{International Financial Reporting Standards (IFRS)} set by IASB (International Accounting Standard Board) is primarily used for this course.

\subsubsection*{Financial Accounting Concepts and Assumptions}
\begin{itemize} \itemsep -2pt
    \item \textbf{Separate Entity Concept:} activities of a business entity are separated from those of the individual owners.
    \item \textbf{Time-period assumption:} Long life of a company can be reported in a shorter time period (annually [fiscal year], quarterly, monthly)
    \item \textbf{Arm's-length transaction assumption:} business dealings between entities are conducted in a rational basis and all parties are acting for their own interests.
    \item \textbf{Cost principle:} financial statement items are measured at their historical costs/original costs at the transaction date.
    The historical cost is assumed to represent the fair market value of the item at the date of transaction.
    \item \textbf{Fair value principle:} the assets and liabilities should be 
    measured at fair value to improve relevance of accounting information.
    \item \textbf{Monetary measurement concept:} Items in financial statements must have value measurable in dollar value.
    \item \textbf{Going concern assumption:} When preparing the financial 
    statements, the business entity is assumed to be able to sustain itself for the foreseeable future (not under liquidation)
\end{itemize}

\subsubsection*{Limitations of Balance Sheet}
\begin{itemize} \itemsep -2pt
    \item Assets are recorded at their purchase cost, not their current market value.
    \item Not all economic assets are included in the balance sheet. 
    (e.g., internally generated brand, good reputation and human talents)
    \item The accounting book value of a company (measured by the amount of equity) is usually less 
    than the company’s market value (measured by market price per share $\times$ number of shares).
\end{itemize}

\subsubsection*{Financial Statements}
\begin{itemize} \itemsep -2pt
    \item Statement of Comprehensive Income (SCI)
    \item Balance Sheet (Statement of Financial Position)\\
    \item Statement of Changes of equity
    \item Statement of Cash Flows
    \item Notes to Financial Statements
\end{itemize}

\subsubsection*{Balance Sheet Items}
\textbf{Assets:} Cash, Accounts Receivable, Inventory, Buildings, Property, Plant and Equipment [Liquidity]\\
\textbf{Liabilities:} Accounts Payable, Income Taxes Payable, Mortgage Payable, Unearned Revenue [Due date]\\
\textbf{Equity:} Capital Stock, Retained Earnings [Dividends distributed reduce retained earnings]\\
\begin{itemize} \itemsep -2pt
    \item $Assets = Liabilities + Equity$
    \item $Assets = \textit{Current Asset} + \textit{Non-Current Assets}$
    \item $Liabilities = \textit{Current Liabilities} + \textit{Non-Current Liabilities}$
    \item $\textit{Retained Earnings(begin)} + \textit{Net Income} - Dividends = \textit{Retained Earnings(end)}$
    \item $\textit{Equity(begin)} + \textit{Increase in Capital Stock} + \textit{Net Income} - Dividends + \textit{Other Comprehensive Income} = Equity(end)$
\end{itemize}

\subsubsection*{SCI Items}
\begin{itemize} \itemsep -2pt
    \item \textit{Expenses = Operating expense $+$ Non-Operating Expense}
    \item $Revenue - Expenses = \textit{Net Income}$\\
    \item $\textit{Net Income} + \textit{Other Comprehensive Income} = \textit{Comprehensive Income}$\\
    \item $\textit{Revenue} - \textit{COGS} = \textit{Gross Profit}$
\end{itemize}

\subsubsection*{Notes to Financial Statements}
\begin{itemize} \itemsep -2pt
    \item Summary of Significant Accounting Policies
    \item Additional Information about the Summary Totals 
    \item Disclosure of Information Not Recognized in the financial statements
    \item Supplementary Information required under the accounting standards
\end{itemize}

\subsection*{Mechanics of Accounting Cycle}
1. Analyse transactions\\
2. Record effects of transaction\\
3. Summarise effects of transactions (journal entries \& trial balance)\\
4. Prepare reports (adjusting entries, financial statements, closing the books)\\

\textbf{Flow of Accounting Records:}\\
Journal Entries, T Accounts, Trial Balance, Income Statement, Statement of Changes in Equity, Balance Sheet\\

\textbf{Criteria for transaction recognition:}\\
1. Involve exchange of resources\\
2. Conducted at Arm's Length between two independent entities\\
3. Can be reliably measured\\

Business Docs such as sales invoices, official receipts, purchases orders, suppliers' invoices capture
business transaction. They provide independent objective evidence of transaction, establish dollar amount, and
facilitate analysis of business events.\\

\subsubsection*{Journal Entry}
A recording of a business transactions; usually includes 
a debit entry and a credit entry to the relevant 
accounts with amount, a date and sometimes an 
explanation of the transaction.\\

\underline{\textbf{Asset Accounts:}}\\
Typically has debit balances.\\DR Asset (Increase) CR Asset (Decrease)\\
\underline{\textbf{Liability or Equity Accounts:}}\\
Typically has credit balances.\\DR Liability/Equity (Decrease) CR Liability/Equity (Increase)\\
\underline{\textbf{Revenue and Expenses Accounts:}}\\
Expenses typically has debit balances. \\Revenue typically has credit balances\\
DR Expense (Increase) CR Expense (Decrease)\\
DR Revenue (Decrease) CR Revenue (Increase)\\
\underline{\textbf{Capital Stock, Retained Earnings, Dividends}}\\
DR Capital Stock (Decrease) CR Capital Stock (Increase)\\
DR Retained Earnings (Decrease) CR Retained Earnings (Increase)\\
DR Dividends (Increase) CR Dividends (Decrease)\\
\underline{\textbf{Personal journal entries notes:}}\\
- Try to think in terms of cash flow\\
- DR Cash -- CR Revenue\\
- DR Expense -- CR Cash\\
- DR Cash -- CR Capital Stock\\
- DR Cash -- CR Revenue; DR COGS -- CR Inventory\\
- DR Notes Payable, Interest Expense -- CR Cash (paying back notes)\\
- DR Dividends -- CR Cash\\
\subsubsection{T-Accounts}
- Basically combine all journal entry transactions by accounts used and calculate ending balances (can be debit or credit).\\
\subsubsection*{Trial Balance}
- Prepare trial balance using the ending balances from \underline{all} T-accounts.\\
- Total debit balance should be \underline{equivalent} to total credit balance.

\subsection*{Accrual Accounting}
System of accounting in which 
revenues are recognized when certain criteria are satisfied 
(i.e., revenues are earned); and expenses are recorded as 
they are incurred to generate the revenue (matching 
principle). This is regardless of \underline{when} cash is paid or 
received.\\
- Revenues can be recognized \textbf{even cash is not received}.\\
- Expenses can be recorded \textbf{even cash is not paid}.\\
- The matching principle requires that all costs and expenses 
incurred in generating revenues must be recognized in the 
same reporting period as the related revenues.\\
- \textbf{Follows the time-period assumption}
which is that the life of a business is divided into distinct 
and relatively short time periods(e.g., 12-month) so that 
accounting information can be timely. 

\subsubsection*{Adjusting entries}
Prepared at the end of each accounting period to adjust accounts to their proper amount on an accrual basis.\\
Do not involve cash. Involves a BL account or SCI account.
\begin{itemize} \itemsep -2pt
    \item Accrued/Unrecorded receivable
    \item Accrued/Unrecorded liabilities
    \item Prepaid expenses and supplies 
    \item Unearned revenue
\end{itemize}

\subsubsection*{Unrecorded receivable (asset)}
\textbf{In Arrears}. Revenues earned during a period that have not been 
recorded by the end of that period. Represent amount of cash or resources to be 
collected by the entity in the future. \\
\textbf{DR Accounts Receivable -- CR Revenue}

\subsubsection*{Unrecorded liabilities (liability)}
Expenses incurred and not paid during a period and are not
recorded at the end of that period. Represents entity's obligation to pay for 
the expense in the future.\\
\textbf{DR Expense -- CR Payable}\\
Example: DR Interest Expense -- CR Interest Payable (Adj Entry)\\
Dr Interest Payable, Loan Payable -- CR Cash 

\subsubsection*{Prepaid Expense (asset)}
Payments made in advance for expense items, e.g., prepaid 
insurance or rent.\\
Journal Entry (day of payment):\\
\textbf{DR Prepaid Expense -- CR Cash}\\
Adjusting Entry (expense \underline{incurred up to date})\\
\textbf{DR Expense -- CR Prepaid Expense}\\
- Calculate fraction of prepaid expense used up at current date\\
- Supplies Example: DR Supplies -- CR Cash; DR Supplies Expense -- CR Supplies;

\subsubsection*{Unearned Revenue (liability)}
Cash amounts received from customers before its 
corresponding revenue can be recognized.\\
Journal Entry (Customer advance payment):\\
\textbf{DR Cash -- CR Unearned Revenue}\\
Adjusting Entry (revenue \underline{earned up to date}):\\
\textbf{DR Unearned Revenue -- CR Revenue}
- Calculate fraction of revenue earned using time\\

\subsubsection*{Closing the books}
Adjust trial balance $\rightarrow$ Prepare SCI $\rightarrow$ Ending Retained Earnings $\rightarrow$ Closing books\\
Making all the Revenue, Expense, Retained Earnings and Dividends account balance to zero.\\
Debit and Credit balances are swapped for these entries.
\begin{itemize} \itemsep -2pt
    \item \textbf{DR Revenue -- CR Expense, Retained Earnings (Net income)}
    \item \textbf{DR Retained Earnings -- CR Dividends}
\end{itemize}


\end{scriptsize}

\end{multicols}

\end{document}