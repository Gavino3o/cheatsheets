%%%%%%%%%%%%%%%%%%%%%%%%%%%%%%%%%%%%%%%%%%%%%%%%%%%%%%%%%%%%%%%%%%%%%%
% writeLaTeX Example: A quick guide to LaTeX
%
% Source: Dave Richeson (divisbyzero.com), Dickinson College
% 
% A one-size-fits-all LaTeX cheat sheet. Kept to two pages, so it 
% can be printed (double-sided) on one piece of paper
% 
% Feel free to distribute this example, but please keep the referral
% to divisbyzero.com
% 
%%%%%%%%%%%%%%%%%%%%%%%%%%%%%%%%%%%%%%%%%%%%%%%%%%%%%%%%%%%%%%%%%%%%%%
% How to use writeLaTeX: 
%
% You edit the source code here on the left, and the preview on the
% right shows you the result within a few seconds.
%
% Bookmark this page and share the URL with your co-authors. They can
% edit at the same time!
%
% You can upload figures, bibliographies, custom classes and
% styles using the files menu.
%
% If you're new to LaTeX, the wikibook is a great place to start:
% http://en.wikibooks.org/wiki/LaTeX
%
%%%%%%%%%%%%%%%%%%%%%%%%%%%%%%%%%%%%%%%%%%%%%%%%%%%%%%%%%%%%%%%%%%%%%%

\documentclass[10pt,landscape]{article}
\usepackage{amssymb,amsmath,amsthm,amsfonts}
\usepackage{multicol,multirow}
\usepackage{calc}
\usepackage{ifthen}
\usepackage{graphicx}
\usepackage{listings}
\usepackage{fancyvrb}
\usepackage[a4paper, margin=4mm, landscape]{geometry}
\usepackage[colorlinks=true,citecolor=blue,linkcolor=blue]{hyperref}


\ifthenelse{\lengthtest { \paperwidth = 11in}}
    { \geometry{top=.3in,left=.3in,right=.3in,bottom=.3in} }
  {\ifthenelse{ \lengthtest{ \paperwidth = 297mm}}
    {\geometry{top=1cm,left=1cm,right=1cm,bottom=1cm} }
    {\geometry{top=1cm,left=1cm,right=1cm,bottom=1cm} }
  }
\pagestyle{empty}
\makeatletter
\renewcommand{\section}{\@startsection{section}{1}{0mm}%
                                {-1ex plus -.5ex minus -.2ex}%
                                {0.5ex plus .2ex}%x 
                                {\normalfont\normalsize\bfseries}}
\renewcommand{\subsection}{\@startsection{subsection}{2}{0mm}%
                                {-1explus -.5ex minus -2ex}%
                                {0.2ex plus -1ex}%
                                {\normalfont\small\bfseries}}
\renewcommand{\subsubsection}{\@startsection{subsubsection}{3}{0mm}%
                                {-1ex plus -.5ex minus -.5ex}%
                                {0.2ex plus -1ex}%
                                {\normalfont\footnotesize\bfseries}}
\makeatother
\setcounter{secnumdepth}{0}
\setlength{\parindent}{0pt}
\setlength{\parskip}{0pt plus 0.5ex}
\graphicspath{ {./images/} }
% -----------------------------------------------------------------------

\title{ACC1701X Finals Cheatsheet AY2023/2024 Semester 2}

\begin{document}

\raggedright
\footnotesize

% \begin{center}
%      \Large{\textbf{ACC1701X Finals Cheatsheet AY2023/2024 Sem 2}} \\
% \end{center}

\begin{multicols}{3}
\setlength{\premulticols}{1pt}
\setlength{\postmulticols}{1pt}
\setlength{\multicolsep}{1pt}
\setlength{\columnsep}{2pt}
\begin{scriptsize}

\begin{center}
  \fbox{%
    \parbox{0.8\linewidth}{\centering \textcolor{black}{
        {\Large\textbf{ACC1701X}}
      \\ \text{AY23/24 SEM 2 (Gavin)}}
    }%
  }
\end{center}

\subsubsection{\underline{Balance Sheet Items and Formulas}}
- Separate Entity Concept, Time-Period Assumption, Arm's Length Transaction Assumption, Cost Principle, Fair Value Principle, Monetary Measurement concept, Going Concern Assumption\\
\begin{itemize} \itemsep -2pt
  \item $Assets = Liabilities + Equity$
  \item $Assets = \textit{Current Asset} + \textit{Non-Current Assets}$
  \item $Liabilities = \textit{Current Liabilities} + \textit{Non-Current Liabilities}$
  \item $\textit{Retained Earnings(begin)} + \textit{Net Income} - Dividends = \textit{Retained Earnings(end)}$
  \item $\textit{Equity(begin)} + \textit{Increase in Capital Stock} + \textit{Net Income} - Dividends + \textit{Other Comprehensive Income} = Equity(end)$
\end{itemize}
\subsubsection{\underline{SCI Items and Formulas}}
\begin{itemize} \itemsep -2pt
    \item \textit{Expenses = Operating expense $+$ Non-Operating Expense}
    \item $Revenue - Expenses = \textit{Net Income}$\\
    \item $\textit{Net Income} + \textit{Other Comprehensive Income} = \textit{Comprehensive Income}$\\
    \item $\textit{Revenue} - \textit{COGS} = \textit{Gross Profit}$
\end{itemize}

\subsubsection{\underline{Sales Discount}}
\textbf{DR Accounts Receivable -- CR Sales Revenue} (Revenue made, credit payment)\\
\textbf{DR Cash, Sales Discount -- Accounts Receivable}

\subsubsection{\underline{Sales Returns and Allowances}}
\textbf{DR Accounts Receivable -- CR Sales Revenue}\\
\textbf{DR COGS -- Inventory}\\
If return required:\\
\textbf{DR Sales Returns and Allowances -- CR Accounts Receivable}\\
\textbf{DR Inventory -- CR COGS}\\

\textit{\textbf{Net Sales} $=$ Sales Revenue $-$ Sales Discount $-$ Sales Returns and Allowances }

\subsubsection{\underline{Treatment of Bad Debt}}
ECL is an \underline{estimated} expense in the SCI. Loss Allowance is a contra-asset account to Accounts Receivable in the balance sheet.\\
\textbf{DR Expected Credit Loss -- CR Loss Allowance}\\
If specific customer identified specifically as uncollectible, \underline{write off}:\\
\textbf{DR Loss Allowance -- CR Accounts Receivable}\\
If written off subsequently pays the outstanding balance, \underline{reinstate}:\\
\textbf{DR Accounts Receivable -- CR Loss Allowance}\\
\textbf{DR Cash -- CR Accounts Receivable}\\
- Loss Allowance (End) = Loss Allowance (Begin) + ECL - Unadjusted Balance\\
- If there is an existing \textbf{credit} balance in loss allowance. Top up less by $ECL - \textit{Unadjusted Balance}$\\
- If there is an existing \textbf{debit} balance in loss allowance. Top up more by $ECL + \textit{Unadjusted Balance}$\\

\textit{Accounts Receivable(net) = Accounts Receivable(begin) $-$ Loss Allowance (end)}\\

\subsubsection{\underline{Notes Receivable}}
\textbf{DR Notes Receivable -- CR Accounts Receivable} (issuing note)\\
\textbf{DR Cash -- CR Notes Receivable, Interest Revenue} (accepting payment)\\
If note is \underline{dishonoured} (convert into AR):\\
\textbf{DR Accounts Receivable -- CR Notes Receivable, Interest Revenue}\\

\subsubsection{\underline{Purchase Discounts/Returns \& Bank Reconcilation \& IC}}
\textbf{DR Inventory -- CR Accounts Payable}\\
\textbf{DR Accounts Payable -- CR Cash, Inventory}(discount amount)\\
- Returns:\\
\textbf{DR Accounts Payable -- CR Inventory}\\
\includegraphics*[height= 2.3cm,width=0.35\linewidth]{bank_reconcilation.png}
\includegraphics*[height= 2.3cm,width=0.55\linewidth]{internal_controls.png}\\

\subsubsection{\underline{Perpetual vs Periodic Inventory System}}
- Periodic: No COGS recorded until end of period. Temporary accounts are used to record the \textbf{purchases, 
freight-in, purchases returns and purchases discounts}.\\
- \textbf{DR: Inventory(Net Purchases), Purchase Returns, Purchase Discounts -- CR: Freight In, Purchases} (Adjust into Inventory Acc)\\
\textit{COGS $=$ Inventory(begin) $+$ Net Purchases $-$ Inventory(end)}\\
\textit{Cost of goods available for sale $=$ Beginning Inv Balance + Net Purchases}\\
- One entry: \textbf{DR COGS -- CR Inventory}\\
- When inventory price rises. Gross margin, net income, inventory(end) is highest with FIFO. WA and LIFO understate inventory(end).

\subsubsection{\underline{Net Realizable Value, NRV}}
If NRV $<$ Cost, write down item by item:\\
\textbf{DR COGS -- CR Allowance for Inventory Write Down (Contra-Inventory, BL)}\\
If NRV $\geq$ Cost, no adjustments needed.\\
\subsection{Liabilities}

\subsubsection{\underline{Payroll and Payroll Related Liabilities}}

- Employers withhold payroll taxes, pensions, insurances, and other deductions to government and agencies.\\
\textbf{DR Salaries Expense -- CR Salaries Payable; Various Payables...}\\
\textbf{DR Salaries Payable -- CR Cash}\\
- Employers contribute to insurance premium and pensions for employees.\\
\textbf{DR Various Expenses... -- CR Various Payables...;}\\
- Employers pay amount withheld for employees and its own contributions.\\
\textbf{DR Various Payables... -- CR Cash}\\
- For this course: 17\% CPF for employer, 20\% CPF for employees. (applied on gross salary)\\
\textbf{DR Salaries Expense -- CR Employees's CPF Payable; Salaries Payable;}\\
\textbf{DR Employer's CPF Expense -- CR Employers CPF Payable}\\
- Employer making payment to CPF Board and Employees.\\
\textbf{DR Salaries Payable -- CR Cash;}\\
\textbf{DR Employers CPF Payable; Employees's CPF Payable -- CR Cash}\\

\subsubsection{\underline{Sales Tax Payable}}

- Sales taxes are paid by customers to the seller, who in turn pays the taxes to the government agency.\\
\textbf{DR Cash -- CR Sales Revenue; Sales Tax Payable}\\

\subsubsection{\underline{GST and VAT}}

$\text{Supplier} \xrightarrow[\text{Input Tax}]{}$ Seller $\xrightarrow[\text{Output Tax}]{}$ Customer\\
\textbf{DR Inventory/Purchases -- CR Cash; GST Input Tax [Purchasing inventory]}\\
\textbf{DR Cash -- CR Sales Revenue; GST Output Tax [Making a sale]}\\
\textbf{DR GST Output Tax -- CR GST Input Tax; GST Tax Payable (Calculated Net GST Payable)}\\
\textbf{DR GST Tax Payable -- CR Cash}\\
- For VAT: Swap out "GST Tax Payable" with "Business Tax Payable"\\

\subsubsection{\underline{Property and Income Tax Payable}}

Property Tax:\\
\textbf{DR Prepaid Property Taxes -- CR Cash}\\
\textbf{DR Property Tax Expense -- CR Prepaid Property Taxes}\\
(Remember to adjust according to months of prepaid used-up)\\
Income Tax:\\
Applied on income before tax in SCI. (Tax rate is 17\% in SG)\\
\textbf{DR Income Tax Expense -- CR Income Tax Payable}\\

\subsubsection{\underline{Provisions and Contingent Liabilities}}

- Provision is reported as an estimated liability on the balance 
sheet. It is recognised when the loss is probable and a
reliable estimate can be made.\\
- Contingent liabilities should be disclosed in notes to financial 
statements if certain conditions are met. NOT reported on the balance sheet.\\
\begin{itemize} \itemsep -2pt
  \item When probability of losses 10-50\%, disclose in notes.
  \item When probability of losses 50\% or more, recognise as provision.
  \item When probability of losses less than 10\%, no need to disclose.
\end{itemize}
\textbf{DR Lawsuit Loss -- CR Lawsuit Provision}\\

\subsubsection{\underline{Provision for Product Warranty}}

\textbf{DR Product Warranty Expense -- Product Warranty Provision}\\
\textbf{DR Product Warranty Provision -- CR Supplies; Wages Payable; Cash etc...}\\
- The balance of warranty liability/provision account does not affect warranty expense.\\

\subsubsection{\underline{Other Revenue and Expenses in SCI}}
- Other Revenue and Expenses (or non-operating income 
and expenses) are items that incurred or earned from 
activities outside of, or peripheral to, the normal 
operations of a firm.\\
- Dividend Revenue, Gain on sale of land, Interest expense etc.\\

\subsection{Property, Plant and Equipment (PPE)}
- PPE Acquisition Cost = Purchase Price + All Costs to get it ready for use.\\
- If PPE is acquired by purchase of two or more assets acquired together at a single price. 
Fair market value is used to calculate the apportionment of lump-sum cost (used in JE).\\

\subsubsection{\underline{Depreciation, DEPR}}
- $\textbf{Depreciation amount} = \text{Cost} - \text{Residual Value}$\\
- Recognised in income statement as operating expense.\\
- Residual value $\rightarrow$ value of asset at end of its useful life.\\ 
- Useful life $\rightarrow$ period over which asset is expected to be used/ number of productions.\\
\textbf{DR Depreciation Expense -- CR Accumulated Depreciation, "PPE"}\\
- Accumulated Depreciation is a contra-asset account to "PPE" account in BL.\\
- $\textbf{Carrying amount} = \text{Cost} - \text{Accumulate Depreciation}$\\
- Change in depreciation estimates only affect future years

\subsubsection{\underline{Straight Line Method of Depreciation}}
- $\textbf{Annual DEPR Expense} = \frac{\text{Cost - Residual Value}}{\text{Estimated Useful Life (Years)}}$\\
- Note: It should be broken into months for end of fiscal year reporting. (Partial Year DEPR)\\

\subsubsection{\underline{Unit of Production Method of Depreciation}}
$\textbf{Annual DEPR Expense} = \frac{\text{Cost - Residual Value}}{\text{Estimated Useful Life (Units)}} \times \text{Units produced}$\\
- Note: If units used in final year do not add up to useful life, carry over.
- Used for natural resources (Depletion). dollar per ton\\
- \textbf{DR Depletion Expense -- CR Accumulated Depletion, "PPE"}\\

\subsubsection{\underline{Declining-Balance Method of Depreciation}}
- Double Declining Balance (DDB) Method:\\
- $\textbf{Depreciation Rate (DDB)} = \frac{1}{\text{Estimated Life (Years)}} \times 2$\\
- Note: Change 2 to 1.5 for 150\% DDB.\\
- $\textbf{Annual DEPR Expense} = \text{Depreciation Rate} \times \text{Remaining Carrying Amount}$\\
- Note: If depreciation amount reduces carrying amount below residual value, reduce to residual value.

\subsubsection{\underline{Impairment of PPE}}
- \textbf{Net fair value} = Fair value - cost of disposal.\\
- \textbf{Recoverable amount} = $\max \{\text{Net fair value, Value in use}\}$\\
- If recoverable amount $<$ carrying amount, recognise impairment loss.\\
- Impairment loss = Carrying amount - Recoverable amount.\\
- Impairment loss is recognised in income statement as non-operating expense.\\
- \textbf{DR Impairment Loss -- CR Accumulated Impairment Losses, "PPE"}\\
- Accumulated Impairment Losses is a contra-asset account to "PPE" account in BL.\\
- Carrying amount in BL = Cost - Accumulated Depreciation - Accumulated Impairment Losses.\\

\subsubsection{\underline{Disposal or Sale of PPE}}
- At disposal, remove 3 accounts: PPE, Accumulated Depreciation, Accumulated Impairment\\
- Gain/Loss = Sales Proceeds - Cost of Disposal - Carrying Amount.\\
- Gain/Loss of disposal is recognised in income statement as non-operating income/expense.\\
- If Gain, credit side. If Loss, debit side.\\
- \textbf{DR Accumulated Depreciation; Accumulated Impairment Loss; Cash -- CR PPE; Gain on Disposal}\\
- \textbf{DR Accumulated Depreciation; Accumulated Impairment Loss; Loss on Disposal -- CR PPE; Cash}\\
- \textbf{DR Accumulated Depreciation; Accumulated Impairment Loss; Cash -- CR PPE}\\

\subsubsection{\underline{Intangible Assets, IA}}
- Patents, trademarks, copyrights, franchises, licences, goodwill.\\
- Internally generated IA are not recognised in BL.\\
- Goodwill: Purchase price - fair market value of net assets acquired.\\
- \textbf{DR Inventory; Long-term operating assets...; Goodwill -- CR Liabilities; Cash}\\

\subsubsection{\underline{Amortisation of IA}}
- Straight-line Amortisation = $\frac{\text{Cost}}{\text{Estimated Useful Life}}$\\
- Intangibles with indefinite useful life are not amortised (goodwill/broadcast licence)\\
- \textbf{DR Amortisation Expense, "IA" -- CR Accumulated Amortisation, "IA"}\\

\subsubsection{\underline{Capitalise vs Expense}}
- Maintenance, repairs, and minor improvements which does not increase productivity are expensed.\\
- \textbf{DR Maintenance Expense -- CR Cash}\\
- Major improvements, extensions, and replacements are capitalised.\\
- \textbf{DR PPE -- CR Cash}\\
- For long term assets, capitalisation can be permanent or limited.\\
- Limited capitalisation is capitalised in separate account and depreciated over useful life.\\
- Freehold land: Not depreciated. Leasehold land: Depreciated over lease term.\\
- R\&D: Research (expensed) $\xrightarrow[]{\text{Tech Feasibility}}$ Development (capitalised)\\
- Targeted advertising: Capitalised if it increases future benefits. General advertising is expensed.\\

\subsection{Equity}

\subsubsection{\underline{Issuance of Shares}}
- Par Value: Legal capital per share. No correlation to Market Value of share\\
- Premium: Amount received above par value. AKA Paid-in capital in Excess of Par\\
- \textbf{DR Cash (Market) -- CR Ordinary Shares (Par); Paid-in Capital in Excess of Par, Ordinary Shares}\\
- \textbf{DR Cash (Market) -- CR Preference Shares (Par); Paid-in Capital in Excess of Par, Preference Shares}\\
- Total Contributed Capital = OS + PS + Paid-in Capital in Excess of Par, OS and PS\\
- If no par value is stated, entire proceeds is credited to shares account.\\
- \textbf{DR Cash -- CR Ordinary Shares}\\
- Non-cash Basis: Use fair value of asset received. If not available, use fair market value of shares issued.\\
- \textbf{DR "Asset" -- CR Ordinary Shares; Paid-in Capital in Excess of Par, Ordinary Shares}\\

\subsubsection{\underline{Treasury Shares}}
- Corporation buy back shares. Treasury Shares account is a contra-equity account. Normally has debit balance.\\
- No dividends will be payable for Treasury shares.\\
- Market price spent to acquire of treasury shares is recorded in Treasury Shares account.\\
- \textbf{DR Treasury Shares -- CR Cash}\\
- Reissuing treasuring share above acquisition cost:
- \textbf{DR Cash (Issued Price) -- CR Treasury Shares (Cost); Paid-in Capital, Treasury Shares (Diff)}\\
- Reissuing treasuring share below acquisition cost (PIC, Treasury Share sufficient balance):\\
- \textbf{DR Cash (Issued Price); Paid-in Capital, Treasury Shares (Diff) -- CR Treasury Shares (Cost)}\\
- Reissuing treasuring share below acquisition cost (PIC, Treasury Share insufficient balance):\\
- \textbf{DR Cash (Issued Price); Paid-in Capital, Treasury Shares (Bal); Retained Earnings (Overflow) -- CR Treasury Shares (Cost)}\\
- Treasury shares amount is deducted from contributed capital + RE in BL, based on acquisition cost.\\
- Paid-in capital, Treasure Shares is \underline{ADDED} to contributed capital in BL.

\subsubsection{\underline{Cash Dividends}}
- \textbf{DR Cash Dividends -- CR Cash Dividends Payable} [Declare]\\
- \textbf{DR Retained Earnings-- CR Cash Dividends} [Close]\\
- \textbf{DR Cash Dividends Payable -- CR Cash} [Pay]\\
- Cash Dividends for Preference Shares: Non-cumulative and Cumulative\\
- Preference Shares: Fixed dividend rate, paid before ordinary shares, use par value.\\
- Non-cumulative: Only current year dividend is paid.\\
- Cumulative: Current year dividend + any unpaid dividend in arrears.\\
- PS and OS dividends are \underline{capped} by Total dividends declared.\\
- \textbf{DR PS Dividends; OS Dividends -- CR Dividends Payable}\\
- \textbf{DR Dividends Payable -- CR Cash}\\

\subsubsection{\underline{Share Dividends}}
- Small share dividends are assigned at market value. (Division: 20-25\% of Total Issued Shares)\\
- \textbf{DR Share Dividends (Market) -- CR Share Dividends Distributable (Par); Paid-in Capital in Excess of Par (Diff)}\\
- Large share dividends are assigned at par value. \\
- \textbf{DR Share Dividends (Par) -- CR Share Dividends Distributable (Par)}\\
- Closing for both:\\
- \textbf{DR Retained Earnings-- CR Share Dividends}\\
- \textbf{DR Share Dividends Distributable -- CR Ordinary Shares}\\

\subsubsection{\underline{Misc}}
- \textbf{Share split:}\\
- Par value is reduced by ratio, number of shares outstanding increased by same ratio.\\
- \textbf{Other Comprehensive Income:}\\
- Exchange differences arising on translation of the equity of foreign subsidaries, unrealised gains/losses of FVTOCI financial assets\\
- Accumulated other Comprehensive Income is added to total contributed capital plus retained earnings.\\

\subsection{Statement of Cash Flows}
- Operating activities: Cash i/o primary business activities.\\
- Investing activities: Cash i/o purchase/sale of long-term assets.\\
- Financing activities: Cash i/o issuance/repurchase of shares, loans, payment of dividends.\\
- Cash received from dividend and interest: OA or IA\\
- Cash paid for interest and dividends: OA or FA\\
- Cash paid for income tax: OA\\
- Cash (begin) + OA + IA + FA = Cash (end)\\
- Ignore non-cash items e.g., depreciation, amortisation and estimated credit loss.\\
- SCF: Operating Act: Net; Investing Act: Net; Financing Act: Net; Net Increase in Cash; Beginning cash balance; End Cash Balance\\

\subsection{Financial Statement Analysis}
- Limitations: Lack of Comparability: differences in accounting classification, accounting 
estimates and methods. Do not contain all relevant information: customer satisfaction, operational data. Historical Data.\\
- Vertical Analysis: Compare financial data within a single period.\\
- Horizontal Analysis: Compare financial data over time.\\
- $\textbf{Percentage of Change} = \frac{\text{Current period amount - Base period amount}}{\text{Base period amount}}  \times 100\%$\\
- $\textbf{Trend Percent} = \frac{\text{Current period amount}}{\text{Base period amount}} \times 100\%$

\subsubsection{\underline{Liquidity Ratios}}
- Use \underline{Year End} Figures.\\
- $\textbf{Current Ratio} = \frac{\text{Current Asset}}{\text{Current Liability}}$\\
- $\textbf{Acid-Test (Quick) Ratio} = \frac{\text{Current asset - Inventories - Prepayments}}{\text{Current Liabilities}}$\\

\subsubsection{\underline{Efficiency Ratios}}
- $\textbf{Fixed asset (PPE) turnover} = \frac{\text{Net Sales}}{\text{Avg Net PPE}}$\\
- $\textbf{Operating Cycle days} = \text{\# Days Sales in Inventory + Avg Collection Period}$ [shorter better]\\
- $\textbf{Purchases Turnover} = \frac{\text{Net Purchases}}{\text{Avg Accounts Payable}}$\\
- $\textbf{Num of Days Purchases in Acc Payable } = \frac{365}{\text{Purchases Turnover}}$\\
- $\textbf{Accounts Receivable Turnover} = \frac{\text{Net Sales}}{\text{Average AR}}$\\
- $\textbf{Average Collection Period} = \frac{365}{\text{Accounts Receivable Turnover}}$\\
- $\textbf{Inventory Turnover} = \frac{\text{COGS}}{\text{Average Inventory}}$ [higher, better]\\
- $\textbf{Num of Days Sales in Inventory} = \frac{365}{\text{Inventory Turnover}}$ [shorter, better]\\

\subsubsection{\underline{Solvency Ratios}}
- $\textbf{Debt Ratio} = \frac{\text{Total Liability}}{\text{Total Asset}}$\\
- $\textbf{Debt-to-Equity Ratio} = \frac{\text{Total Liability}}{\text{Total Equity}}$\\
- $\textbf{Time Interest Earned} = \frac{\text{Income bfr interest and taxes (Operating Profit)}}{\text{Annual Interest Expense}}$ [higher better]\\

\subsubsection{\underline{Profitability Ratios}}
- $\textbf{Profit Margin (Return on Sales)} =\frac{\text{Net Income}}{\text{Net Sales}}$\\
- $\textbf{Return of Assets} = \frac{\text{Net Income}}{\text{Avg Total Assets}}$\\
- $\textbf{Asset Turnover} = \frac{\text{Net Sales}}{\text{Avg Total Assets}}$\\
- $\textbf{Earnings per share (EPS)} = \frac{\text{Net Income - Preference Dividends}}{\text{Avg Num of Outstanding Ordinary Shares}}$\\
- $\textbf{Price-Earnings (PE) Ratio} = \frac{\text{Market Value of Shares}}{\text{Net Income}} = \frac{\text{Price per share}}{\text{Earnings per share}}$\\

\subsubsection{\underline{Cash Flow Ratios}}
- $\textbf{Cash flow to Net Income Ratio} = \frac{\text{Cash Flow from Operations}}{\text{Net Income}}$\\
- $\textbf{Cash flow Adequacy Ratio} = \frac{\text{Cash Flow from Operations}}{\text{Cash Paid for Capital Expenditure}}$\\

\subsubsection{\underline{DuPont Framework}}
- $Profitability \times Efficiency \times Leverage$\\
- $\textbf{Return of Equity} \newline = \text{Return on Sales} \times \text{Asset Turnover} \times \text{Assets-to-equity Ratio}$\\
$ = \frac{\text{Net Income}}{\text{Net Sales}} \times \frac{\text{Net Sales}}{\text{Avg Total Assets}} \times \frac{\text{Avg Total Assets}}{\text{Avg Total Equity}}$\\
$ = \frac{\text{Net Income}}{\text{Avg Total Equity}}$


\end{scriptsize}

\end{multicols}

\end{document}